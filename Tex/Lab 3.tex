\documentclass[11pt]{article}

\usepackage{fancyhdr} 
\usepackage{lastpage}
\usepackage{extramarks} 
\usepackage{graphicx} 
\usepackage{enumitem}
\usepackage{mcode}
\usepackage{graphicx}
\usepackage{hyperref}
\usepackage{float}
\usepackage[parfill]{parskip}
\usepackage{amsmath}

% Margins
\topmargin=-0.45in
\evensidemargin=0in
\oddsidemargin=0in
\textwidth=6.5in
\textheight=9.0in
\headsep=0.25in 

\linespread{1} % Line spacing

% Custom macros

% Set up the header and footer
\pagestyle{fancy}
\lhead{\hmwkClass: \hmwkTitle} % Top left header
\chead{} % Top center header
\rhead{\hmwkAuthorName \firstxmark} % Top right header
\lfoot{\lastxmark} % Bottom left footer
\cfoot{} % Bottom center footer
\renewcommand\headrulewidth{0.4pt} % Size of the header rule

\setlength\parindent{0pt} % Removes all indentation from paragraphs
   
%----------------------------------------------------------------------------------------
%	NAME AND CLASS SECTION
%----------------------------------------------------------------------------------------

\newcommand{\hmwkTitle}{Lab 3} % Assignment title
\newcommand{\hmwkDueDate}{12/8/2016} % Due date
\newcommand{\hmwkClass}{ENEE 222} % Course/class
\newcommand{\hmwkAuthorName}{} % Your name

%----------------------------------------------------------------------------------------
%	TITLE PAGE
%----------------------------------------------------------------------------------------

\title{
\vspace{2in}
\textmd{\textbf{\hmwkClass:\ \hmwkTitle}}\\
\normalsize\vspace{0.1in}\small{Due\ on\ \hmwkDueDate}\\
\vspace{0.1in}\large{\textit{\hmwkClassInstructor\ \hmwkClassTime}}
\vspace{3in}
}

\author{\textbf{\hmwkAuthorName}}
\date{} % Insert date here if you want it to appear below your name

%----------------------------------------------------------------------------------------

\begin{document}

% \maketitle


\section{Abstract}



\section{Body}

\begin{enumerate} 
\item We begin by record the phrase 'I love Matlab':
\begin{lstlisting} 
fs = 10000;
r = audiorecorder(fs,16,1); % Record at fs = 10000
recordblocking(r, 8);
\end{lstlisting}

\item And saving the speech as a '.wav' file. Note this process is repeated for any other sound recording file used.
\begin{lstlisting}
audiowrite('holum.wav', r.getaudiodata, fs); % Save data as .wav
\end{lstlisting}


\item We used the templates provided online as a template for constructing a method of determining voiced, unvoiced, and silenced portions of a recording that used both energy levels and zero-crossings. In particular, we defined a function that took a recording, sample rate, zero-crossing threshold, energy threshold, and a silence energy threshold, that separated the given vector in three equally sized vectors containing the results.

The method is documented in: 
\begin{verbatim}
https://www.asee.org/documents/zones/zone1/2008/student/ASEE12008_0044_paper.pdf
\end{verbatim}

\begin{lstlisting}
zc_threshold = 0.100; % zero_crossings / num_samples
en_threshold = 0.075; % Energy threshold
silence_en_threshold = 0.03; % Silence energy threshold

sound_file = 'holum.wav';
[speech, fs] = audioread(sound_file);

% Identify voiced, unvoiced, and silence from basic recording
[voiced, unvoiced, silence, t] = part_3(speech, fs, zc_threshold, 
		en_threshold, silence_en_threshold);

hold off;
plot(t, voiced, 'b');
hold on;
plot(t, unvoiced, 'r');
plot(t, silence, 'g');
legend('voiced', 'unvoiced', 'silence')
xlabel('time (s)');
ylabel('Amp');
title('Plot of Voiced/Unvoiced/Silence for Standard Recording');
\end{lstlisting}

\item Adding white noise to the speech, then repeating section three:

\begin{lstlisting}
[sp, fs] = audioread('holum.wav');
noisy = AddNoise(sp, 5);

sound_file = 'noisy_speech.wav';
[speech, fs] = audioread(sound_file);

zc_threshold = 1; % zero_crossings / num_samples
en_threshold = 0.075; % Energy threshold
silence_en_threshold = 0.03; % Silence energy threshold

[voiced, unvoiced, silence, t] = part_3(speech, fs, zc_threshold, 
	en_threshold, silence_en_threshold);

hold off;
plot(t, voiced, 'b');
hold on;
plot(t, unvoiced, 'r');
plot(t, silence, 'g');
legend('voiced', 'unvoiced', 'silence')
xlabel('time (s)');
ylabel('Amp');
title('Plot of Voiced/Unvoiced/Silence for Noisy Recording');
\end{lstlisting}


\item Obtaining the spectrum of speech in short segments.


\item Locating unvoiced segments lost in the noisy signal


\item Labeling voiced/unvoiced segments based on spectrum


\item Moving average filters


\item Repeating part 3 on a filtered noise sample

\end{enumerate}



\section{Conclusion}



\section{Discussion}



\section{Code}

\begin{itemize}

\item Part-3 function for picking and returning voiced, unvoiced, and silenced portions of an audio sample.

\begin{lstlisting}
function [voiced, unvoiced, silence, t] = part_3(speech, 
		fs, zc_threshold, en_threshold, silence_en_threshold)
% This computes voiced, unvoiced, and silence arrays passed in through the
% Specified limits

    close all;
    clear plot;

    x = 1:length(speech);
    t = x./fs;

    f_size = 128;
    len = length(speech);
    num_F = floor(len/(f_size));
    beg = 1;
    en = f_size;

    for i = 1:num_F
        speech_frame = speech(beg:en);

        % Compute zero crossings and energy for the frame
        zc = zero_crossings(speech_frame);
        theta = sum(abs(speech_frame))/length(speech_frame);
        energ(beg:en) = theta;
        crossing(beg:en) = zc;

        % Check for zc threshold
        ts = zc/f_size <= zc_threshold;
        es = theta >= en_threshold;
        voiced_i(beg:en) = (ts && es);

        % 'Silence' is low energy
        silence_i(beg:en) = theta < silence_en_threshold;

        % Check for energy threshold
        beg = beg + f_size;
        en = en + f_size;
    end

    [voiced, unvoiced] = split_vectors(speech, voiced_i); 
    [silence, unvoiced] = split_vectors(unvoiced, silence_i);

end
\end{lstlisting}


\item Zero-crossing Function used in part 3.
\begin{lstlisting}
function num_crossings = zero_crossings(xx)
% Count and return the number of 0 crossings in the vector, xx.

num_crossings = 0;
for i = 1:length(xx) - 1
    num_crossings = num_crossings + abs(.5 * sign(xx(i)) - .5 * sign(xx(i+1)));
end
\end{lstlisting}

\end{itemize}

\end{document}